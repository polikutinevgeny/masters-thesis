\documentclass[12pt, unicode]{beamer}

\usepackage{fontspec}
\usepackage{polyglossia}
\setdefaultlanguage{russian}
\setotherlanguage{english}
\setsansfont{Fira Sans}
\newfontfamily\cyrillicfont{Fira Mono}
\renewcommand\UrlFont{\ttfamilylatin}

\usepackage{nicefrac}

\usepackage{lmodern}
\usepackage{graphicx}
\usepackage{multicol}
\usepackage{subcaption}
\usepackage{array}
\usepackage{calc}
\usepackage{colortbl}
\usepackage{tikz}
\usetikzlibrary{positioning,decorations,calc}
\graphicspath{{./images/}}

\newcolumntype{C}[1]{>{\centering\arraybackslash}m{#1}}

\newif\ifstartcompletesineup
\newif\ifendcompletesineup
\pgfkeys{
    /pgf/decoration/.cd,
    start up/.is if=startcompletesineup,
    start up=true,
    start up/.default=true,
    start down/.style={/pgf/decoration/start up=false},
    end up/.is if=endcompletesineup,
    end up=true,
    end up/.default=true,
    end down/.style={/pgf/decoration/end up=false}
}
\pgfdeclaredecoration{complete sines}{initial}
{
    \state{initial}[
    width=+0pt,
    next state=upsine,
    persistent precomputation={
        \ifstartcompletesineup
        \pgfkeys{/pgf/decoration automaton/next state=upsine}
        \ifendcompletesineup
        \pgfmathsetmacro\matchinglength{
            0.5*\pgfdecoratedinputsegmentlength / (ceil(0.5* \pgfdecoratedinputsegmentlength / \pgfdecorationsegmentlength) )
        }
        \else
        \pgfmathsetmacro\matchinglength{
            0.5 * \pgfdecoratedinputsegmentlength / (ceil(0.5 * \pgfdecoratedinputsegmentlength / \pgfdecorationsegmentlength ) - 0.499)
        }
        \fi
        \else
        \pgfkeys{/pgf/decoration automaton/next state=downsine}
        \ifendcompletesineup
        \pgfmathsetmacro\matchinglength{
            0.5* \pgfdecoratedinputsegmentlength / (ceil(0.5 * \pgfdecoratedinputsegmentlength / \pgfdecorationsegmentlength ) - 0.4999)
        }
        \else
        \pgfmathsetmacro\matchinglength{
            0.5 * \pgfdecoratedinputsegmentlength / (ceil(0.5 * \pgfdecoratedinputsegmentlength / \pgfdecorationsegmentlength ) )
        }
        \fi
        \fi
        \setlength{\pgfdecorationsegmentlength}{\matchinglength pt}
    }] {}
    \state{downsine}[width=\pgfdecorationsegmentlength,next state=upsine]{
        \pgfpathsine{\pgfpoint{0.5\pgfdecorationsegmentlength}{0.5\pgfdecorationsegmentamplitude}}
        \pgfpathcosine{\pgfpoint{0.5\pgfdecorationsegmentlength}{-0.5\pgfdecorationsegmentamplitude}}
    }
    \state{upsine}[width=\pgfdecorationsegmentlength,next state=downsine]{
        \pgfpathsine{\pgfpoint{0.5\pgfdecorationsegmentlength}{-0.5\pgfdecorationsegmentamplitude}}
        \pgfpathcosine{\pgfpoint{0.5\pgfdecorationsegmentlength}{0.5\pgfdecorationsegmentamplitude}}
    }
    \state{final}{}
}

\tikzset{
    between/.style args={#1 and #2}{
        at = ($(#1)!0.5!(#2)$)
    }
}

\tikzset{
    between base/.style args={#1 and #2}{
        between=#1.base and #2.base
    }
}

\setbeamertemplate{note page}[plain]
\setbeameroption{show notes on second screen=right}

\newif\ifmetropolis
\metropolistrue

\title[Система автоматизированного A/B тестирования]{Система автоматизированного A/B~тестирования}
\author[Поликутин Е.Ю.]{
    \vbox{\raggedright%
        Студент группы М9119-09.04.01иибд\\ 
        Поликутин Евгений Юрьевич%
    }
    \vskip 20pt%
    \indent\vbox{\raggedright%
        Руководитель:\\
        Консультант:\\
        Старший аналитик данных ООО <<Амаяма Авто>>\\
        Олейников Игорь Сергеевич\ifmetropolis\else\vskip -0.5cm\fi%
    }
}
\date{11 июня 2021 г.}

\newif\ifPS
%\PStrue

\ifmetropolis
    \usetheme[progressbar=frametitle,numbering=fraction]{metropolis}
    \makeatletter
    \setlength{\metropolis@progressinheadfoot@linewidth}{2pt}
    \makeatother
\else
    \usetheme{CambridgeUS}
\fi

\setbeamersize{text margin left=3.5mm,text margin right=3.5mm} 

\newcommand{\pa}[1]{\left(#1\right)}

\usepackage{alphalph}
\makeatletter
\newcommand{\makeAlph}[1]{(\alphalph{\arabic{#1}})}
\makeatother

\newcommand*\rot{\rotatebox{90}}

\tolerance=1
\emergencystretch=\maxdimen
\hyphenpenalty=10000
\hbadness=10000

\begin{document}
    
    \frame{\thispagestyle{empty}\titlepage}
    
    \note{Защищается студент группы М9119-09.04.01иибд Поликутин Евгений Юрьевич по теме Система автоматизированного A/B тестирования. Консультант --- старший аналитик данных ООО <<Амаяма Авто>> Олейников Игорь Сергеевич}
    
    \begin{frame}[fragile]{A/B тестирование}
    	\begin{figure}[h]
   			\centering
   			\includegraphics[width=0.9\textwidth]{ab_testing.png}
   			\caption{Принцип A/B тестирования}
    	\end{figure}
    \end{frame}

	\note{A/B тестирование --- это статистический эксперимент, проводящийся путём разделения пользователей на 2 и более группы, представления им различных вариантов продукта и сравнением ключевых показателей. A/B тесты широко применяются как в веб-разработке, так и в разработке мобильных приложений.}
	
	\begin{frame}[fragile]{Дром}
		\begin{figure}[h]
			\centering
			\includegraphics[width=0.9\textwidth]{drom.png}
			\caption{Дром}
		\end{figure}
	\end{frame}

	\note{GGGGG}
    
    \begin{frame}[fragile]{Анализ существующих решений}
    	\scriptsize
    	\color{black}
		\begin{table}[h]
			\centering
			\begin{tabular}{|p{4cm}|c|c|c|c|c|c|c|c|c|}
			\hline
			& \rot{Google\ } & \rot{Optimizely\ } & \rot{VWO\ } & \rot{Evergage\ } & \rot{Taplytics\ } & \rot{Countly\ } & \rot{Matomo\ } & \rot{Planout\ } & \rot{Sixpack\ }\\
			\hline
			мобильные приложения + веб & 
			\textpm & + & + & + & + & + & + & \textminus & \textminus \\
			\hline
			самостоятельная доработка системы &
			\textminus & \textminus & \textminus & \textminus & \textminus & + & + & + & + \\
			\hline
			анализ сырых данных &
			+ & \textpm & \textpm & + & + & \textpm & \textpm & \textminus & \textminus \\
			\hline
			гибкая настройка метрик &
			\textminus & \textminus & \textminus & \textminus & \textminus & \textminus & \textminus & \textminus & \textminus \\
			\hline 
			хостинг on-premises &
			\textminus & \textminus & \textminus & \textminus & + & + & + & + & + \\
			\hline
			корректный непрерывный мониторинг тестов &
			+ & + & + & + & + & + & \textminus & \textminus & \textminus \\
			\hline
			методики уменьшения дисперсии &
			\textminus & \textminus & \textminus & \textminus & \textminus & \textminus & \textminus & \textminus & \textminus \\
			\hline
			работа с существующими базами данных &
			\textminus & \textminus & \textminus & \textminus & \textminus & \textminus & \textminus & \textminus & \textminus \\
			\hline
			\end{tabular}
			\caption{Сравнительный анализ аналогичных решений}
		\end{table}
    \end{frame}

	\note{Большинство присутствующих на рынке программных продуктов для A/B тестирования являются облачными решениями для аналитики, требующими передачи данных на сервера других компаний; такой формат также не позволяет самостоятельно дорабатывать систему и добавлять новые метрики. Существующие же on-premises решения обладают слабым функционалом, а используемые СУБД не позволяют обрабатывать необходимый объём событий. Решения с исходным кодом позволяют только разделять пользователей на когорты, не содержат функционала анализа результатов.}
    
    \begin{frame}[fragile]{Звуковое поле}
        \begin{block}{Уравнение Гельмгольца}
            \ifmetropolis
                \smallskip
            \fi
            \begin{equation}\label{eq::3DH}
                \pa{\rho\pa{z}\nabla\cdot\pa{\frac{1}{\rho\pa{z}}\nabla} +  K^2\pa{x,y,z}}p\pa{x,y,z}=-\delta\pa{x}\delta\pa{y}\delta\pa{z-z_s}
            \end{equation}
            \begin{align*}
                K\pa{z}&=\frac{\omega}{c\pa{z}}&&K\pa{z}=\frac{\omega}{c\pa{z}}\pa{1+i\eta\beta}
            \end{align*}
            \centering
            \begin{tabular}{ll}
                 $\rho\pa{z}$ --- плотность среды & $c\pa{z}$ --- скорость звука\\
                 $\omega=2\pi f$ --- циклическая частота & $\beta$ --- коэффициент затухания\\
                 $\eta=\nicefrac{1}{40\pi\log_{10}e}$&
            \end{tabular}
        \end{block}
    \end{frame}

	\note{Звуковое поле, создаваемое точечным источником в трёхмерном волноводе мелкого моря, описывается трёхмерным уравнением Гельмгольца. Здесь $x,y$ --- горизонтальные координаты, $z$ --- глубина (положительная величина), $\rho\pa{z}$ --- плотность. Может быть рассмотрено два случая: с учётом затухания волн и без. От этого зависит выбор вида коэффициента $K$. $c\pa{z}$ --- скорость звука на глубине $z$. $\omega$ --- циклическая частота ($f$ -- частота источника), $\beta$ --- коэффициент затухания, $\eta$ --- вот такое странное число.}

    \begin{frame}[fragile]{Модовое разложение поля}
        \begin{block}{Модовое разложение}
            \ifmetropolis
                \smallskip
            \fi
            \begin{equation}
                p\pa{x,y,z}=\sum\limits_{j=1}^JA_j\pa{x,y}\varphi_j\pa{z,x,y}
            \end{equation}
        \end{block}
        \begin{block}{Уравнение горизонтальной рефракции}
            \ifmetropolis
                \smallskip
            \fi
            \begin{equation}
                \frac{\partial^2 A_j}{\partial x^2} + \frac{\partial^2 A_j}{\partial y^2}+k_j^2 (x,y)A_j=-\varphi_j(z_s)\delta(x)\delta(y)
            \end{equation}
        \end{block}
        \begin{block}{Псевдодифференциальное МПУ}
            \ifmetropolis
                \smallskip
            \fi
            \begin{equation}
                A_j\pa{x,y}=e^{k_{j,0}x}\mathcal{A}_j\pa{x,y}
            \end{equation}
            \begin{equation}
                \frac{\partial\mathcal{A}_j\pa{x,y}}{\partial x}=ik_{j,0}\pa{\sqrt{1+L_j}-1}\mathcal{A}_j\pa{x,y}
            \end{equation}
            \begin{equation}
                k_{j,0}^2L_j=\frac{\partial^2}{\partial y^2}+k_j^2\pa{x,y}-k_{j,0}^2\nonumber
            \end{equation}
        \end{block}
    \end{frame}

	\note{Решение $p\pa{x,y,z}$ уравнения Гельмгольца может быть выражено в форме модового разложения, где $A_j$ --- модовые амплитуды, а $\varphi_j$ ---модовые функции. Модовые аплитуды удовлетворяют уравнению горизонтальной рефракции, где $z_s$ --- глубина источника. Модовые функции $\varphi_j\pa{z,x,y}$ и соответствующие им горизонтальные числа $k_j(x,y)$ могут быть получены из решения акустической спектральной задачи. Таким образом появляется возможность рассматривать уравнение для каждой моды отдельно. Учитывая звуковые волны, распространяющиеся в положительном направлении оси $x$, вводя относительное волновое число $k_{j,0}$ и исключая главную осцилляцию получим псевдо-дифференциальное модовое параболическое уравнение.}

    \begin{frame}[fragile]{Аппроксимация оператора квадратного корня}
        \begin{block}{Аппроксимация Паде}
            \ifmetropolis
                \smallskip
            \fi
            \begin{equation}
                F\pa{\lambda}\approx\mathcal{R}\pa{F,l,m}\pa{\lambda}\equiv\frac{P^F_{l,m}\pa{\lambda}}{Q^F_{l,m}\pa{\lambda}}
            \end{equation}
        \end{block}
        \begin{block}{Линеаризованное МПУ}
            \ifmetropolis
                \smallskip
            \fi
            \begin{equation}
                \frac{\partial\mathcal{A}_j\pa{x,y}}{\partial x}=\frac{P_{l,m}\pa{L_j}}{Q_{l,m}\pa{L_j}}\mathcal{A}_j\pa{x,y}
            \end{equation}
        \end{block}
        \begin{block}{Дискретизированное МПУ}
            \ifmetropolis
                \smallskip
            \fi
            \begin{equation}
                \begin{gathered}
                    \mathcal{A}_j^{b+1,q}=\pa{a_{l,m}^0+\sum\limits_{i=1}^m\frac{a^i_{l,m}}{1+b^i_{l,m}L_j^\delta}}\mathcal{A}_j^{n,q}\\
                    1\leqslant l\leqslant m
                \end{gathered}
            \end{equation}
        \end{block}
    \end{frame}

    \note{Для решения псевдо-дифференциальное модовое параболическое уравнения необходимо выполнить линеаризацию оператора квадратного корня. Для этого может быть использована аппроксимация Паде. Порядок этой аппроксимации определяет широкоугольные свойства решения. Далее дискретизация оператора $L_j^\delta$ производится с использованием конечных разностей}

    \begin{frame}[fragile]{Split-step Pad\'e}
        \begin{block}{Дискретизация по $x$}
            \ifmetropolis
                \smallskip
            \fi
            \begin{equation}
                \mathcal{A}_j^{n+1}=e^{ik_{j,0}h\pa{\sqrt{1+L_j}-1}}\mathcal{A}_j^{n}
            \end{equation}
        \end{block}
        \begin{block}{Аппроксимация экспоненты}
            \ifmetropolis
                \smallskip
            \fi
            \begin{equation}
                e^{ik_{j,0}h\pa{\sqrt{1+L_j}-1}}\approx\frac{\tilde{P}_{l,m}\pa{L_j}}{\tilde{Q}_{l,m}\pa{L_j}}=\tilde{a}_{l,m}^0+\sum\limits_{i=1}^m\frac{\tilde{a}_{l,m}^i}{1+\tilde{b}_{l,m}^iL_j}
            \end{equation}
        \end{block}
        \begin{block}{Дискретизированное МПУ}
            \ifmetropolis
                \smallskip
            \fi
            \begin{equation}
                \mathcal{A}_j^{n+1,q}=\pa{\tilde{a}_{l,m}^0+\sum\limits_{i=1}^m\frac{\tilde{a}_{l,m}^i}{1+\tilde{b}_{l,m}^iL_j^\delta}}\mathcal{A}_j^{n,q}
            \end{equation}
        \end{block}
        \vfill\null
        \footnotetext[1]{Collins M.D. A split-step Pad\'e solution for parabolic equation method // The Journal of the Acoustical Society of America --- 1993 --- Т. 93, № 4. --- С. 1736---1742}
    \end{frame}

    \note{Также существует другой подход, изначально предложенный Коллинзом. В его основе лежит смена порядка дискретизации и применения аппроксимации Падэ. При достаточно малом шаге $h$ псевдодифференциальное МПУ может быть формально решено в виде. Затем аппроксимация Падэ применяется к экспоненте. Полученное уравнение совпадает с дискретизацией методом Крэнка-Николсон с точностью до значений коэффициентов, и дискретизация оператора $L_j^\delta$ может быть также выполнена конечными разностями}
    
    \begin{frame}[fragile]{Граничные условия}
        \begin{figure}[t]
            \begin{tikzpicture}
                \coordinate(C11);
                \foreach \i in {2,...,7} {
                    \pgfmathtruncatemacro\index{\i-1}
                    \coordinate[right of=C1\index](C1\i);
                }
                \foreach \i in {2,...,4} {
                    \pgfmathtruncatemacro\index{\i-1}
                    \coordinate[below of=C\index1](C\i1);
                    \foreach \j in {2,...,7} {
                        \pgfmathtruncatemacro\index{\j-1}
                        \coordinate[right of=C\i\index](C\i\j);
                    }
                }
                \draw (C11) -- (C17);
                \draw (C11) -- (C41);
                \draw (C17) -- (C47);
                \draw[dashed] (C12) -- (C42);
                \draw[dashed] (C16) -- (C46);
                \draw[decorate, decoration={complete sines}] (C41) -- (C47);
                
                \node[above=0 of C12](Y0){$y_0$};
                \node[above=0 of C16](Y1){$y_1$};
                
                \node[above=0 of C11]{$-\varepsilon$};
                \node[above=0 of C17]{$+\varepsilon$};
                
                \node[between base=C24 and C34]{\Large $\Omega$};
                \node[between base=C21 and C32]{PML};
                \node[between base=C26 and C37]{PML};
                
                \coordinate[at=($(Y0.base)!0.4!(Y1.base)$)](YA1);
                \coordinate[at=($(Y0.base)!0.6!(Y1.base)$)](YA2);
                \draw[->] (YA1) -- (YA2) node[midway,yshift=2mm]{$y$};
                
                \coordinate[left=2mm of C11](X0);
                \coordinate[left=2mm of C41](X1);
                
                \coordinate[at=($(X0.base)!0.4!(X1.base)$)](XA1);
                \coordinate[at=($(X0.base)!0.6!(X1.base)$)](XA2);
                \draw[->] (XA1) -- (XA2) node[midway,xshift=-2mm]{$x$};
            \end{tikzpicture}
        \end{figure}
        \begin{block}{Оператор в PML области}
            \ifmetropolis
                \smallskip
            \fi
            \begin{equation}
                k_{j,0}^2L_j^{PML}=\frac{1}{1+i\beta\pa{y}}\frac{\partial}{\partial y}\frac{1}{1+i\beta\pa{y}}\frac{\partial}{\partial y}+k_j^2+k_{j,0}^2
            \end{equation}
            \begin{equation}
                \beta\pa{y}=\beta_0\pa{\frac{\left|y-y_b\right|}{\varepsilon}}^3=\beta_0\zeta^3\equiv\beta\pa{\zeta}\,,\quad\zeta\in\left[0, 1\right]
            \end{equation}
        \end{block}
        \footnotetext[2]{Berenger J.-P. A perfectly matched layer for the absorbtion of electromagnetic waves // Journal of Computational Physics --- 1994 --- Т. 114, № 2. --- С. 185---200}
    \end{frame}

    \note{Одной из особенностей МПУ является то, что их решение всегда рассмат­ривается в неограниченной области, поэтому её искусственное ограничение является обязательным при численном решении МПУ. В данной работе были использованы PML граничные условия, которые впервые были использованы Беренджером для уравнений Максвелла. В основе метода лежит расширение вычислительной области с целью плавного поглощения волн исходящих из неё, для этого оператор $L_j$ заменяется на оператор $L_j^{PML}$, где $\beta\pa{y}$ монотонная функция, возрастающая при движении вглубь PML слоёв и равная нулю в области $\Omega$. В рамках данной работы была использована кубическая функция, где $y_b$ --- соответствующая граница области, а $\beta_0$ максимальное значение функции. При этом на новых границах ставятся обычные нулевые условия Дирихле}

    \begin{frame}[fragile]{Лучевая теория распространения звука}
        \begin{block}{}
            \ifmetropolis
                \smallskip
            \fi
            \begin{equation}
                \mathcal{A}_j\pa{x,y}=M_j\pa{x,y}e^{ik_{j,0}S_j\pa{x,y}}+o\pa{\nicefrac{1}{k_{j,0}}}
            \end{equation}
        \end{block}
        \begin{block}{Уравнение Гамильтона-Якоби}
            \ifmetropolis
                \smallskip
            \fi
            \begin{equation}
                \pa{\frac{\partial S_j\pa{x,y}}{\partial x}}^2+\pa{\frac{\partial S_j\pa{x,y}}{\partial y}}^2=n_j\pa{x,y}
            \end{equation}
            \begin{equation*}
                n_j\pa{x,y}\equiv \nicefrac{k_j\pa{x,y}}{k_{j,0}}
            \end{equation*}
        \end{block}
        \begin{block}{Гамильтонова система}
            \ifmetropolis
                \smallskip
            \fi
            \begin{equation}
                \begin{aligned}
                    \frac{dx_j\pa{l}}{dl}&=\frac{\xi_j\pa{l}}{n_j\pa{x,y}}\qquad&\frac{d\xi_j\pa{l}}{dl}&=\frac{\partial n_j\pa{x,y}}{\partial x}\\
                    \frac{dy_j\pa{l}}{dl}&=\frac{\eta_j\pa{l}}{n_j\pa{x,y}}\qquad&\frac{d\eta_j\pa{l}}{dl}&=\frac{\partial n_j\pa{x,y}}{\partial y}\\
                \end{aligned}
            \end{equation}
        \end{block}
    \end{frame}

	\note{Предполагая, что волновые числа $k_j\pa{x,y}$ являются медленно изменяющейся функцией, решение уравнения с использованием лучевой теории распространения звука может быть выражено в следующей форме, где функция $M_j\pa{x,y}$ --- амплитуда нулевого порядка, а функция $S_j\pa{x,y}$ называется эйконалом и может быть найдена из уравнения Гамильтона-Якоби, где $n_j\pa{x,y}\equiv \nicefrac{k_j\pa{x,y}}{k_{j,0}}$ --- индекс горизонтальной рефракции. Решение этого уравнения связано с решением так называемой Гамильновой системы, где $l$ является натуральным параметром, обозначающим длину кривой вдоль траектории распространения луча, а $\xi,\eta$ сопряжённые переменные к $x,y$ --- момент. Из решения этой системы определяются траектории горизонтальных лучей распространения звука.}

    \begin{frame}{Лучевые начальные условия}
        \begin{block}{Начальное условие}
            \ifmetropolis
                \smallskip
            \fi
            \begin{gather}
                \mathcal{A}_j\pa{x_0,y}=M_j\pa{x_0,y}e^{ik_{j,0}S_j\pa{x_0,y}}\,,\qquad y_0\leqslant y\leqslant y_1\\
                S_j\pa{l}=\int\limits_0^ln_j\pa{l}dl\\
                M_j\pa{l}=\frac{M_{j,0}}{n_j\pa{l}}\sqrt{\frac{\cos\alpha}{\nicefrac{\partial y\pa{l,\alpha}}{\partial\alpha}}}\\
                M_{j,0}=\frac{e^\frac{i\pi}{4}}{\sqrt{8\pi k_{j,0}}}\nonumber
            \end{gather}
        \end{block}
        \vspace{-0.5cm}
        \begin{block}{Простые начальные условия}
            \ifmetropolis
                \smallskip 
            \fi
            \begin{equation}
                x\pa{l}=l\cos\alpha\,,\quad y\pa{l}=l\sin\alpha\,,\quad S\pa{l}=l\,,\quad M_{j,0}\pa{l}=\frac{M_0}{\sqrt{x\pa{l}^2+y\pa{l}^2}}
            \end{equation}
        \end{block}
    \end{frame}

	\note{При решении очень широкоугольных уравнений требуются соответствующие широкоугольные начальные условия, чтобы избежать численного шума. Наиболее часто используемые начальные условия Гаусса и Грина создают численный шум даже при использовании первого члена аппроксимации. Для борьбы с численным шумом могут быть использованы лучевые начальные условия. Начальное условие рассматривается на расстоянии $x_0$ от источника, сравнимым с длиной волны. Начальное условие вычисляется с использованием лучевой теории звука, при этом предполагается, что среда не изменяется с координатой $x$. Так как $x_0$ обычно достаточно мало, можно в дальнейшем также предположить независимость от $y$, что приводит к упрошенным начальным условиям. Несмотря на то, что упрощённые условия могут быть использованы в широком классе задач, при сильном изменении среды около источника необходимы более общие условия.}

    \begin{frame}{Импульс звукового сигнала}
        \begin{block}{Значение импульса в частотной области}
            \ifmetropolis
                \smallskip
            \fi
            \begin{equation}
                \hat{I}_r\pa{\xi}=\overline{\hat{P}\pa{x_r,y_r,z_r,\xi}\cdot e^{-i\tau\omega\pa{\xi}}}
            \end{equation}
            \begin{equation}
                \hat{P}\pa{x,y,z,\xi}=p\pa{x,y,z,f\pa{\xi}}\cdot\overline{\hat{g}}\pa{\xi}
            \end{equation}
        \end{block}
    	\begin{block}{Sound Exposure Level}
			\begin{equation}
				SEL\pa{x,y,z,f_1,f_2}=10\log\pa{\int\limits_{f_1}^{f_2}\left|\hat{P}\pa{x,y,z,\xi\pa{f}}\right|^2df}
			\end{equation}
    	\end{block}
    \end{frame}

	\note{Также может быть рассмотрена задача вычисления импульса звукового сигнала в произвольных точках среды. Пусть источник излучает сигнал $g\pa{t}$. Тогда импульс $I$ в источнике может быть вычислен в спектральной (частотной) области. Здесь $p$ это решение уравнения Гельмгольца для частоты $f$, $\tau$ --- время, которое требуется звуку для достижения приёмника. Также может быть вычислен Sound Exposure Level, который является интегралом модуля квадрата звукового давления при диапазоне рассматриваемых частот $f_1, f_2$}

	\begin{frame}[fragile]{Обзор существующих решений}
		\begin{block}{Решение уравнения Гельмгольца}
			\begin{itemize}
				\item BELLHOP
				\item Traceo3D
				\item Закрытые программные продукты Океанографического института в Вудс-Хоуле и центральной школы Лиона
				\item Различного рода внутренние adhoc скрипты
			\end{itemize}
		\end{block}
	\end{frame}

	\note{На данный момент существует несколько программных продуктов позволяющих вычислять численное решение уравнения Гельмгольца. BELLHOP и Traceo3D, основанные на методе суммирования Гауссовых пучков и лучевой теории распространения звука соответственно. Недостатком этих методов является использование геометроакустического приближения, которое является недостаточно точным при моделировании источников звука, имеющих частоту менее 1 кГц. Океанографический институт в Вудс-Хоуле и центральная школа Лиона имеют закрытые комплексы программ, основанные на решении трёхмерного параболического уравнения, однако решение таких уравнений требует запредельных затрат памяти и времени, расчёт самых простых задач занимает не менее суток}

	\begin{frame}{Программная реализация}
		\begin{block}{}
			\begin{itemize}
				\item Язык C++17
				\item Репозиторий: {\footnotesize \url{https://github.com/GoldFeniks/AMPLE}}
				\item Зависимости
				\begin{itemize}
					\item Boost C++
					\item ALGLIB
					\item fftw3
					\item nlohmann::json
                    \item Eigen
				\end{itemize}
				\item Модули
				\begin{itemize}
					\item CAMBALA\qquad {\footnotesize \url{https://github.com/Nauchnik/Acoustics-at-home/}}
					\item DORK\qquad {\footnotesize \url{https://github.com/GoldFeniks/DORK}}
					\item delaunay\qquad {\footnotesize \url{https://github.com/GoldFeniks/delaunay}}
					\item zip\qquad {\footnotesize \url{https://github.com/GoldFeniks/zip}}
				\end{itemize}
			\end{itemize}
		\end{block}
	\end{frame}

	\note{Для выполнения поставленных задач была написана программа на языке С++17. Репозиторий с кодом размешен на гитхабе. На данный момент программа имеет следующие зависимости, репозитории CMABALA, DORK, delaunay и zip используются в качестве сабмодулей}

	\begin{frame}{Входные и выходные данные}
		\begin{block}{Входные данные}
			\begin{itemize}
				\item config.json
				\item Дополнительные текстовые или бинарные файлы
				\vspace{-0.25cm}
				\begin{multicols}{2}
					\begin{itemize}
						\item Модовые функции и собственные значения
						\item Частоты
						\item Батиметрия
					\end{itemize}
					\columnbreak
					\begin{itemize}
						\item Гидрология
						\item Координаты приёмников
						\item Функция и спектр источника
					\end{itemize}
				\end{multicols}
			\end{itemize}
		\end{block}
		\begin{block}{Выходные данные}
			\begin{itemize}
				\item meta.json
				\item config.json
				\item Файлы вывода
				\item Папки вывода нескольких фалов {\small\ttfamilylatin <job\_type>/meta.json}
			\end{itemize}
		\end{block}
	\end{frame}

	\note{Входным файлом программы является файл в конфигурации в формате JSON. В файле задаются параметры среды, в том числе различные пространственные данные, функция и спектр источника, которые могут быть сохранены как в самом конфигурационном файле, так и в дополнительных текстовых или бинарных файлах. Результатом работы программы являются файлы в формате JSON, описывающие параметры выходных данных, их размерности и пути к ним, а также информацию о модах в источнике, временах прихода сигнала и времени работы программы. Выходные данные могут иметь бинарный или текстовый формат, а также содержаться в папках, если задача требует вывода при разных частотах, каждая такая папка содержит свой файл meta.json, описывающий данные}
    
    \begin{frame}[fragile]{Вычислительные эксперименты. Волновод Пекериса}
        \begin{figure}[h]
            \centering
            \begin{subfigure}[t]{0.35\textwidth}
                \centering
                \includegraphics[width=\textwidth]{pekeris.pdf}
                \caption{Аналитическое решение}
            \end{subfigure}
            \begin{subfigure}[t]{0.35\textwidth}
                \centering
                \includegraphics[width=\textwidth]{pekeris_wampe.pdf}
                \caption{Решение ШМПУ}
            \end{subfigure}\\
            \begin{subfigure}[t]{0.35\textwidth}
                \centering
                \includegraphics[width=\textwidth]{pekeris_n5.pdf}
                \caption{SSP, $p=5$}
            \end{subfigure}
            \begin{subfigure}[t]{0.35\textwidth}
                \centering
                \includegraphics[width=\textwidth]{pekeris_n17.pdf}
                \caption{SSP, $p=17$}
            \end{subfigure}
            \caption{Акустическое поле в волноводе Пекериса}
        \end{figure}
    \end{frame}

    \note{Было проведено несколько численных экспериментов. В качестве первого эксперимента было проведено моделирование распространения звука в волноводе с постоянной глубиной дна. Полученное решение сравнивалось с аналитическим решением и решением ШМПУ с использованием аппроксимации Клаербоута квадратного корня и начальных условий Грина. Как видно из рисунка использование SSP метода с простыми лучевыми начальными условиями и большим порядком аппроксимации позволяет получить решение, которое почти идеально аппроксимирует аналитическое}
    
    \begin{frame}[fragile]{Вычислительные эксперименты. Волновод Пекериса}
        \begin{figure}[h]
            \centering
            \includegraphics[width=0.7\textwidth]{pekeris_pml_n13.pdf}
            \caption{Акустическое поле в волноводе Пекериса на глубине $z=30\ \text{м.}$, слои PML отмечены красной пунктирной линией, ширина слоёв составляет $1$ километр, порядок аппроксимации Падэ $p=13$}
        \end{figure}
    \end{frame}

    \note{Принцип действия PML граничных условий изображён на рисунке 2, так волны, исходящие из вычислительной области постепенно затухают при движении вглубь поглощающего слоя}

    \begin{frame}[fragile]{Вычислительные эксперименты. Подводный каньон}
        \vspace{-0.25cm}
        \begin{figure}[h]
            \centering
            \includegraphics[width=0.3\textwidth]{canyon_transparent.png}
            \caption{Схематичное изображение волновода}
        \end{figure}
        \vspace{-0.75cm}
        \begin{figure}[h]
            \centering
            \includegraphics[width=0.75\textwidth]{canyon_n11.pdf}
            \caption{Акустическое поле в волноводе с подводным каньоном}
        \end{figure}
    \end{frame}

	\note{Далее был рассмотрен подводный каньон. Решение было получено с использованием 11 членов аппроксимации и простых лучевых начальных условий. Как видно из рисунка, полученное решение имеет апертуру почти $\pm 90^\circ$}
    
    \begin{frame}[fragile]{Вычислительные эксперименты. Подводный каньон}
        \begin{figure}[h]
            \centering
            \includegraphics[width=0.9\textwidth]{canyon_compare.pdf}
            \caption{Сравнение результатов вычисления акустического поля в мелком море с подводным каньоном вдоль оси $y=0\ \text{км.}$ на глубине $z=10\ \text{м}$.}
        \end{figure}
    \end{frame}

    \note{Полученное решение также сравнивалось с решением трёхмерного параболического уравнения вдоль оси $y=0$ на глубине $10$ метров. Из рисунка видно, что решения достаточно сильно совпадают, при этом решение МПУ требует значительно меньше вычислительных ресурсов}

	\begin{frame}{Вычислительные эксперименты. Подводный каньон}
		\begin{figure}[h]
			\centering
            \begin{subfigure}{0.45\textwidth}
                \centering
                \includegraphics[width=\textwidth]{canyon_rays_1.pdf}
                \caption{1-ая мода}
            \end{subfigure}
            \begin{subfigure}{0.45\textwidth}
                \centering
                \includegraphics[width=\textwidth]{canyon_rays_2.pdf}
                \caption{2-ая мода}
            \end{subfigure}\\
            \begin{subfigure}{0.45\textwidth}
                \centering
                \includegraphics[width=\textwidth]{canyon_rays_3.pdf}
                \caption{3-ая мода}
            \end{subfigure}
            \begin{subfigure}{0.45\textwidth}
                \centering
                \includegraphics[width=\textwidth]{canyon_rays.pdf}
                \caption{4-ая мода}
            \end{subfigure}
			\caption{Лучи, соответствующие вертикальным модам}
		\end{figure}
	\end{frame}

	\note{Каньон захватывает звук, поэтому лучи образуют петли при достаточно небольшом угле отклонения от главной оси распространения. С увеличением номера моды захват становится более выраженным}
    
    \begin{frame}[fragile]{Вычислительные эксперименты. Клиновидный волновод}
        \vspace{-0.25cm}
        \begin{figure}[h]
            \centering
            \includegraphics[width=0.3\textwidth]{wedge_transparent.png}
            \caption{Схематичное изображение волновода}
        \end{figure}
        \vspace{-0.75cm}
        \begin{figure}[h]
            \centering
            \begin{subfigure}[t]{0.35\textwidth}
                \centering
                \includegraphics[width=\textwidth]{wedge_wampe.pdf}
                \caption{Решение ШМПУ}
            \end{subfigure}
            \begin{subfigure}[t]{0.35\textwidth}
                \centering
                \includegraphics[width=\textwidth]{wedge_n13.pdf}
                \caption{SSP}
            \end{subfigure}
            \caption{Акустическое поле в клиновидном волноводе}
        \end{figure}
    \end{frame}

    \note{Далее было рассмотрено моделирование распространения звука в клиновидном волноводе. Результаты вычисления изображены на рисунке 8}
    
    \begin{frame}[fragile]{Вычислительные эксперименты. Клиновидный волновод}
        \begin{figure}[h]
            \centering
            \vspace{-0.25cm}
            \begin{tikzpicture}
                \node (IMG) at (0,0) {\includegraphics[width=0.6\textwidth]{wedge_comp_1.pdf}};
                \node[left=.5cm of IMG]{\textbf{(a)} $x=\ 1\ \text{км.}$};
            \end{tikzpicture}
            \begin{tikzpicture}
                \node (IMG) at (0,0) {\includegraphics[width=0.6\textwidth]{wedge_comp_9.pdf}};
                \node[left=.5cm of IMG]{\textbf{(b)} $x=\ 9\ \text{км.}$};
            \end{tikzpicture}
            \begin{tikzpicture}
                \node (IMG) at (0,0) {\includegraphics[width=0.6\textwidth]{wedge_comp_17.pdf}};
                \node[left=.45cm of IMG]{\textbf{(c)} $x=25\ \text{км.}$};
            \end{tikzpicture}
            \caption{Сравнение результатов вычисления акустического поля вдоль $x$}
        \end{figure}
    \end{frame}

    \note{Было произведено сравнения акустического поля на различных горизонтах $x$. Из рисунка видно, что вблизи источника SSP решение не образует численного шума, а при отдалении от источника становится заметна более широкая апертура этого решения.}
    
    \begin{frame}[fragile]{Вычислительные эксперименты. Клиновидный волновод}
        \begin{figure}[h]
            \centering
            \begin{subfigure}[t]{0.55\textwidth}
                \centering
                \includegraphics[width=\textwidth]{wedge_comp.pdf}
            \end{subfigure}
            \hfill
            \begin{subfigure}[t]{0.55\textwidth}
                \centering
                \includegraphics[width=\textwidth]{wedge_comp_close.pdf}
            \end{subfigure}
            \caption{Сравнение результатов вычисления акустического поля вдоль $y$}
        \end{figure}
    \end{frame}

    \note{Также было произведено сравнение с решением методом изображений. Решения всех методов почти совпадают, не смотря на адиабатическую природу модовых параболических уравнений, при этом большая апертура SSP метода сильнее приближает решение к решению методом изображений вдали от источника.}

    \begin{frame}{\normalsize Вычислительные эксперименты. Волновод с реальной батиметрией}
        \vspace{-0.25cm}
   		\begin{figure}
   			\centering
   			\includegraphics[width=0.4\textwidth]{sakhalin_transparent.png}
   			\caption{Рельеф дна}
   		\end{figure}
        \vspace{-0.5cm}
   		\begin{figure}
   			\centering
   			\includegraphics[width=0.7\textwidth]{sound_profile.pdf}
   			\caption{Скорость звука}
   		\end{figure}
    \end{frame}

	\note{Также была рассмотрена задача с настоящими данными батиметрии и скорости звука в воде, представленными на слайде}

	\begin{frame}[fragile]{\normalsize Вычислительные эксперименты. Волновод с реальной батиметрией}
		\begin{figure}[h]
			\centering
            \begin{subfigure}{.6\textwidth}
                \centering
                \includegraphics[width=\textwidth]{sakhalin_wampe_z4.pdf}
                \caption{Решение ШМПУ}
            \end{subfigure}
            \begin{subfigure}{.6\textwidth}
                \centering
                \includegraphics[width=\textwidth]{sakhalin_n11_z4.pdf}
                \caption{SSP}
            \end{subfigure}
			\caption{Акустическое поле источника}
		\end{figure}
	\end{frame}

	\note{Звуковое поле было вычислено на глубине $z=4\ \text{м.}$. Из рисунка видно, что решение ШМПУ существенно уступает решению, полученному методом SSP с большим порядком аппроксимации Падэ. Также можно заметить, как звук фокусируется в области с большей глубиной.}

    \begin{frame}{Ход работы}
        \begin{block}{Сделано}
            \begin{itemize}
                \item SSP метод решения уравнения с использованием аппроксимации Падэ произвольного порядка
                \item Трассировка лучей, соответствущих модам
                \item Лучевые начальные условия
                \item Вычисление импульса звукового сигнала
                \item Вычисление SEL
                \item Расчёт решения, зависящего от $x,y,z$
            \end{itemize}
        \end{block}
        \begin{block}{Возможные улучшения}
            \begin{itemize}
                \item Учёт пространственных неоднородностей скорости звука в воде и дне
                \item Учёт межмодового взаимодействия
            \end{itemize}
        \end{block}
    \end{frame}

	\note{Таким образом, в рамках проделанной работы были исследованы методы решения МПУ с использованием аппроксимации Падэ произвольного порядка, лучевых начальных условий и PML граничных условий, разработана численная схема их решения, разработан комплекс программ на языке C++, реализующий полученную численную схему с использованием пакета CAMBALA и возможностью вычисления импульса звукового сигнала, значения распределения уровней SEL и координат распространения лучей, соответствующих вертикальным модам; проведены различные вычислительные эксперименты и изучена корректность и применимость полученного метода в сравнении с ШМПУ и другими методами решения уравнения Гельмгольца.}
    
\end{document}