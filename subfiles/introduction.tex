\documentclass[../document.tex]{subfiles}

\begin{document}
	\par\gls{abtesting} применяется в продуктовом цикле разработки во многих современных IT-компаниях для проверки гипотез и анализа влияния вносимых изменений на показатели продукта. Это важный процесс, который де-факто обеспечивает обратную связь между изменениями и командой разработки.
	\par <<Дром>> --- автомобильный портал, который является <<доской объявлений>> о продаже автомобилей. <<Дром>> также использует А/Б тестирование для указанных выше целей в процессе разработки всех продуктов компании.
	\par На текущий момент такие эксперименты планируются и проводятся аналитиками вручную, что представляет собой достаточно большой объём работы и отнимает большое количество их рабочего времени. Ручная работа также несёт за собой потенциальные ошибки, которые могут оставаться невыявленными в течении долгого времени, или не выявлены вообще. На основании же этих ошибок могут приниматься новые неверные решения. Более того, процесс принятия решений на основе A/B тестов сейчас слишком гибок, и часто гипотезы рождаются и проверяются уже post-hoc, что неверно методологически.
	\par Автоматизация A/B тестов в данной ситуации позволит унифицировать процесс принятия решений, так как планирование тестов будет осуществляться до их запуска, снизить нагрузку на аналитиков --- переложив проведение тестов на менеджеров продуктов и уменьшить количество потенциальных ошибок, так как человеческий фактор будет частично исключён. Она также позволит ускорить проведение A/B тестов, что напрямую влияет на скорость разработки новых функций продуктов и позволяет более эффективно расходовать ресурсы команд.
	\par Целью работы является разработка и внедрение системы автоматизированного A/B тестирования в компании <<Дром>>.
	\par Задачи работы:
	\begin{enumerate}
		\item Провести обзор предметной области;
		\item Провести обзор существующих решений, разработать спецификации нового решения;
		\item Реализовать и внедрить разработанное решение.
	\end{enumerate}
	\par Разрабатываемая система выгодно отличается от существующих решений тем, что может быть развёрнута \gls{onpremises} и не требует передачи конфиденциальной информации в облако, где к ней потенциально могут получить доступ третьи лица. Она так же может использовать существующие базы данных компании, из-за чего отсутствуют жёсткие требования к ним. Также система использует современные методы последовательного и множественного тестирования, уменьшения дисперсии, и не ограничена в выборе метрик. Решение также позволяет наблюдать за процессом прохождения теста, что даёт возможность рано выявить ошибки и остановить тест по достижению статистической значимости.
	\par В главе 1 проведён обзор предметной области;
	\par В главе 2 проведён обзор и сравнительный анализ существующих решений, разработаны спецификации предлагаемой системы;
	\par В главе 3 описаны методы и результаты реализации системы, итоги внедрения и практического применения.
\end{document}