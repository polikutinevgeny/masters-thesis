\documentclass[../document.tex]{subfiles}

\begin{document}
	\subsection{Описание проекта}
	\par Система состоит из двух взаимосвязанных частей: API вместе с графическим интерфейсом в виде веб-приложения отвечает за создание, управление и просмотр результатов A/B тестов; исполнитель регулярно проверяет в базе наличие тестов, требующих расчёта результатов, проводит вычисления и записывает результаты в базу данных. Исполнитель и API разворачиваются отдельно, в виде двух контейнеров в составе одного пода Kubernetes.
	\par Средством реализации системы был выбран язык программирования Python и библиотеки FastAPI. Веб-интерфейс был реализован на языке TypeScript с применением фреймворка React.
	\par Статистический тест bootstrap mSPRT выполнен в виде модуля для Python на языке C++ с помощью библиотеки Boost.Python в связи с большими вычислительными затратами и вытекающей из них фактической невозможностью выполнять их на Python.
	\par Для работы с базой данных использовалась ORM Sqlalchemy, работа с данными проводилась с помощью библиотеки pandas. Также использовались различные статистические функции библиотеки statsmodels.
	\par Структура модулей системы приведена на рисунке \ref{image:module_plot}.
	\begin{figure}[H]
		\caption{\label{image:module_plot}Структура модулей системы}
	\end{figure}
	\par Структура классов системы приведена на рисунке \ref{image:class_plot}.
	\begin{figure}[H]
		\caption{\label{image:class_plot}Структура классов системы}
	\end{figure}
	\par Диаграмма состояний теста приведена на рисунке \ref{image:test_state_plot}.
	\begin{figure}[H]
		\caption{\label{image:test_state_plot}Диаграмма состояний A/B теста}
	\end{figure}
	\par Метрики и стратегии распределения по когортам, как самые часто изменяемые элементы системы, были реализованы в виде автоматически регистрирующихся плагинов, регистрация которых происходит во время запуска. Это позволяет добавлять новые метрики и стратегии, не внося изменения в остальную систему.
	\par Также реализована автоматическая генерация формы создания A/B теста исходя из метаданных плагинов, опять же, для облегчения добавления новых плагинов. Это делается с помощью библиотеки pydantic, генерирующей JSONSchema для параметров плагинов, библиотеки react-jsonschema-form, создающей форму по схеме, и доработки генерации схемы для поддержки условных выражений.
	\subsection{Реализация и тестирование}
	\subsubsection{Краткие характеристики системы}
	\par Объём работы составил <ЧИСЛО> строк (<ЧИСЛО> килобайт) на ЯП Python, <ЧИСЛО> строк (<ЧИСЛО> килобайт) --- на C++ и <ЧИСЛО> строк (<ЧИСЛО> килобайт) --- на Typescript.
	\par Количество модулей составило <ЧИСЛО> штук, графический интерфейс содержит 4 страницы.
	\subsubsection{Проблемы при реализации}
	\subsection{Практическое применение и внедрение}
\end{document}