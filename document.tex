%% !TEX program = xelatex
\documentclass{fefu}
\usepackage{subfiles}
% \setmainlanguage[babelshorthands=true]{russian}
\newfontfamily\cyrillicfont{Times New Roman}[Script=Cyrillic]

% All necessary packages must be defined here
\usepackage{mathtools}
\usepackage{float}
\usepackage{minted}
\usepackage{dirtree}
\usepackage{multirow}
\usepackage{textcomp}
\usepackage{tabu}
\usepackage[
	separate-uncertainty = true,
	multi-part-units = repeat
	]{siunitx}
\usepackage{graphicx}
\usepackage{placeins}

\renewcommand{\listingscaption}{Код программы}
\newcommand*\rot{\rotatebox{90}}

\setglossarystyle{index}
\renewcommand*{\glsnamefont}[1]{\textmd{#1} ---}
\makeatletter
\renewcommand{\@idxitem}{\parindent1.25cm\par\indent}
\makeatother
\makeglossaries

\newglossaryentry{abtesting}
{
	name=A/B тестирование,
	description={процесс проведения рандомизированного эксперимента путем разделения пользователей на 2 (или более) группы, представления им различных вариантов продукта и сравнением результатов с помощью заранее выбранной метрики}
}

\newglossaryentry{onpremises}
{
	name=on-premises,
	description={программное обеспечение, которое разворачивается на оборудовании, принадлежащем компании}
}

\newglossaryentry{saas}{name={SaaS},description={программное обеспечение, которое предоставляется по регулярной подписке, обычно расположенное на серверах компании-разработчика}}
\newglossaryentry{ctr}{name={CTR},description={метрика отношения количества кликов к количеству просмотров}}

\author{Поликутин Евгений Юрьевич}
\setgroup{М9119-09.04.01иибд}
\setfaculty{09.04.01 Информатика и вычислительная техника}
\setprogram{Искусственный интеллект и большие данные}
\setsupervisor{Шевченко Игорь Иванович}{доцент ШЕН ДВФУ, к.т.н.}
\setdirector{Мирин Илья Геннадьевич}{}
\setsecretary{Тихонова Татьяна Сергеевна}{}
\setconsultant{Олейников Игорь Сергеевич}{}
\setdeputy{Сапрыкина Елена Валерьевна}{к.э.н.}
\setreviewer{Поречин Дмитрий Владимирович}{аналитик ООО <<Фарпост ИТ>>}
\setschool{цифровой экономики}
\title{Система автоматизированного A/B тестирования}
\setpracticetitle{Преддипломная практика}
\setplace{Школе Цифровой Экономики}
\setexportsupervisor{Сапрыкина Елена Валерьевна}{зам. директора ШЦЭ}

\urlstyle{same}

\addbibresource{references.bib}
%\newcommand{\macro}[2]{\texttt{\textbackslash#1\{#2\}}}

\addto\captionsrussian{%
	\renewcommand{\figurename}{Рисунок}%
}

\begin{document}
	%\maketitle{practice}
    \maketitle{thesis}
    \makethesistitlebackside[1.2]
    \begin{thesistask}[Поликутину Евгению Юрьевичу]{24.12.20}{124-01-03-18}{05.07.21}{23.12.20}
    	\taskitem \uline{разработать и внедрить на проекте <<Дром>> систему автоматизированного A/B тестирования}
    	\taskitem \uline{
    		-исследование существующих продуктов для A/B тестирования\\
    		-исследование существующих статистических методов, применяемых для A/B тестирования\\
    		-разработка системы, автоматизирующей A/B тестирование на проекте <<Дром>>\\
    		-внедрение разработанной системы в бизнес-процесс проекта <<Дром>>
    	}
    	\taskitem \uline{
    		-Johari R., Pekelis L., Walsh D. Always Valid Inference: Bringing Sequential Analysis to A/B Testing (2019)\\
    		-Johari R., Koomen. P., Pekelis L., Walsh D. Peeking at A/B Tests: Why It Matters, and What to Do about It (2017)\\
    		-Kohavi R., Tang D., Xu Y. Trustworthy Online Controlled Experiments: A Practical Guide to A/B Testing (2020)
    	}
    \end{thesistask}
    \begin{abstract}
    	\subfile{subfiles/annotation.tex}
    \end{abstract}
    \tableofcontents
    \section*{Введение}
    \subfile{subfiles/introduction.tex}
    \newpage
    \printglossary[type=main,title={Глоссарий}]
    \newpage
    \section{A/B тестирование: обзор и постановка задачи}
    \subfile{subfiles/analysis.tex}
    \newpage
    \section{Спецификации системы}
    \subfile{subfiles/specifications.tex}
    \newpage
    \section{Подробности реализации и результаты внедрения}
    \subfile{subfiles/implementation.tex}
    \newpage
    \section*{Заключение}
    \subfile{subfiles/conclusion.tex}
    \newpage
    \printbibliography
    \newpage
    \begin{supervisorreview}[Поликутина Евгения Юрьевича]{отлично}{90}{}
    	\par Выпускная квалификационная работа Евгения Юрьевича Поликутина выполнена по теме автоматической обработки А/B тестов с использованием различных инструментов математической статистики, позволяющих добиться более высокой точности и скорости выполнения тестов. Выбранное Евгением направление на настоящий момент является очень актуальным в области продуктовой разработки и является неотъемлемой частью системы непрерывной разработки программных продуктов. Изначально А/B – тесты проводились группами исследователей различных лекарственных средств и медицинских препаратов для доказательства эффективности и безопасности тех или иных лекарств. С развитием технических возможностей обратной связи между разработчиками и пользователями онлайн приложений, подобные медицине техники стало возможно воспроизводить при разработке приложений для смартфонов, web-сайтов и подобных им полностью цифровых продуктов.
    	\par ООО <<Амаяма Авто>> разрабатывает множество различных программных продуктов от флагманского Дром.Авто до менее известных, таких как Дром.Штрафы и Номерограм. Не все проекты являются коммерческими, например Дром.ПДД является бесплатным.
    	\par Целью автоматизированной системы А/B-тестирования было обеспечение продуктовых менеджеров различных проектов понятным инструментом, позволяющим им оценивать эффективность каждого изменения в курируемом продукте в виде измеримых показателей. Основными проблемами в любом А/B-тесте являются: необходимость ждать достаточное время для принятия решения о применении или отклонении сделанного изменения и требование не принимать решение на основании промежуточных результатов. Подобные системы, разумеется, разрабатываются в том числе и компаниями гигантами, самый яркий пример — Firebase от Google. Главным недостатком подобных облачных систем является требование выгрузки в них, зачастую, конфиденциальных для бизнеса данных. Также важным является точность проведения самого теста, ведь от результата будет зависеть принимаемое менеджером решение и допустить слишком большую долю ложноположительных или ложноотрицательных результатов аналитик не имеет права.
    	\par Евгений Юрьевич последовательно взялся за работу и, помимо реализации технической части A/B-тестирования, провел несколько опытов с ускоренным тестированием, последовательным тестированием с помощью mSPRT сложных гипотез, ввел поправки на множественное тестирование, а также определение победившего варианта в случае невозможности определения полного порядка на графе вариантов.
    	\par Важно отметить, что магистрант Поликутин решил не только бизнес-проблему конкретного предприятия, но и предложил несколько стабильных подходов, позволяющих определять победителей в статистических опытах и научных работах в которых применяются статистические гипотезы и статистические тесты. Известно, что многие научные работы в области естественных наук грешат непоследовательностью именно в области статистических тестов для проверок основной гипотезы (например, не применяют поправки на множественное тестирование, либо останавливают тестирование по достижению минимально-значимых результатов). 
    	\par Работа выполнялась Евгением Юрьевичем Поликутиным самостоятельно, в координации с научным руководителем и консультантом от предприятия и была внедрена в процесс непрерывной разработки внутренних проектов Дрома как система помогающая принятию решений продакт-менеджеров и аналитиков данных. На системе, разработанной Евгением было проведено свыше 75 реальных А/Б- тестов, показавших полностью статистически корректные и воспроизводимые результаты. 
    	\par Текст работы состоит из трех глав, введения и заключения. В первой главе проводится обзор математических методов, используемых при решении подобных задач в настоящее время и доступных из открытых источников. Вторая глава посвящена описанию разработанной системы и применением в ней ранее описанных методов. В третьей главе рассматриваются результаты реального применения системы на практике и делается вывод о необходимости и полезности применения данной системе при работе с проверкой гипотез статистическими методами.
    	\par Работа Евгения Юрьевича Поликутина выполнена полностью в соответствии с поставленной задачей, последовательно, грамотно изложены как процесс разработки системы, так и результаты его внедрения. Считаю, что работа соответствует всем требованиям, предъявляемым к магистерским ВКР, заслуживает оценки <<отлично>>, а студент Поликутин заслуживает присвоения степени «магистр» по соответствующему направлению подготовки.
    \end{supervisorreview}
    \begin{review}[Поликутина Евгения Юрьевича]{отлично}{}
    	РЕЦЕНЗИЯ
    \end{review}
\end{document}

