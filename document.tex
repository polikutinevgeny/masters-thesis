%% !TEX program = xelatex
\documentclass{fefu}
\usepackage{subfiles}
% \setmainlanguage[babelshorthands=true]{russian}
\newfontfamily\cyrillicfont{Times New Roman}[Script=Cyrillic]

% All necessary packages must be defined here
\usepackage{mathtools}
\usepackage{float}
\usepackage{minted}
\usepackage{dirtree}
\usepackage{multirow}
\usepackage{textcomp}
\usepackage{tabu}
\usepackage[
	separate-uncertainty = true,
	multi-part-units = repeat
	]{siunitx}
\usepackage{graphicx}
\usepackage{placeins}

\renewcommand{\listingscaption}{Код программы}
\newcommand*\rot{\rotatebox{90}}

\setglossarystyle{index}
\renewcommand*{\glsnamefont}[1]{\textmd{#1} ---}
\makeatletter
\renewcommand{\@idxitem}{\parindent1.25cm\par\indent}
\makeatother
\makeglossaries

\newglossaryentry{abtesting}
{
	name=A/B тестирование,
	description={процесс проведения рандомизированного эксперимента путем разделения пользователей на 2 (или более) группы, представления им различных вариантов продукта и сравнением результатов с помощью заранее выбранной метрики}
}

\newglossaryentry{onpremises}
{
	name=on-premises,
	description={программное обеспечение, которое разворачивается на оборудовании, принадлежащем компании}
}

\newglossaryentry{saas}{name={SaaS},description={программное обеспечение, которое предоставляется по регулярной подписке, обычно расположенное на серверах компании-разработчика}}
\newglossaryentry{ctr}{name={CTR},description={метрика отношения количества кликов к количеству просмотров}}

\author{Поликутин Евгений Юрьевич}
\setgroup{М9119-09.04.01иибд}
\setfaculty{09.04.01 Информатика и вычислительная техника}
\setprogram{Искусственный интеллект и большие данные}
\setsupervisor{Шевченко Игорь Иванович}{доцент ШЕН ДВФУ, к.т.н.}
\setdirector{Мирин Илья Геннадьевич}{}
\setsecretary{Тихонова Татьяна Сергеевна}{}
\setconsultant{Олейников Игорь Сергеевич}{}
\setdeputy{Сапрыкина Елена Валерьевна}{к.э.н.}
\setreviewer{Поречин Дмитрий Владимирович}{аналитик ООО <<Фарпост ИТ>>}
\setschool{цифровой экономики}
\title{Система автоматизированного A/B тестирования}
\setpracticetitle{Преддипломная практика}
\setplace{Школе Цифровой Экономики}
\setexportsupervisor{Сапрыкина Елена Валерьевна}{зам. директора ШЦЭ}

\urlstyle{same}

\addbibresource{references.bib}
%\newcommand{\macro}[2]{\texttt{\textbackslash#1\{#2\}}}

\addto\captionsrussian{%
	\renewcommand{\figurename}{Рисунок}%
}

\begin{document}
	%\maketitle{practice}
    \maketitle{thesis}
    \makethesistitlebackside[1.2]
    \begin{thesistask}[Поликутину Евгению Юрьевичу]{24.12.20}{124-01-03-18}{05.07.21}{23.12.20}
    	\taskitem \uline{разработать и внедрить на проекте <<Дром>> систему автоматизированного A/B тестирования}
    	\taskitem \uline{
    		-исследование существующих продуктов для A/B тестирования\\
    		-исследование существующих статистических методов, применяемых для A/B тестирования\\
    		-разработка системы, автоматизирующей A/B тестирование на проекте <<Дром>>\\
    		-внедрение разработанной системы в бизнес-процесс проекта <<Дром>>
    	}
    	\taskitem \uline{
    		-Johari R., Pekelis L., Walsh D. Always Valid Inference: Bringing Sequential Analysis to A/B Testing (2019)\\
    		-Johari R., Koomen. P., Pekelis L., Walsh D. Peeking at A/B Tests: Why It Matters, and What to Do about It (2017)\\
    		-Kohavi R., Tang D., Xu Y. Trustworthy Online Controlled Experiments: A Practical Guide to A/B Testing (2020)
    	}
    \end{thesistask}
    \begin{abstract}
    	\subfile{subfiles/annotation.tex}
    \end{abstract}
    \tableofcontents
    \section*{Введение}
    \subfile{subfiles/introduction.tex}
    \newpage
    \printglossary[type=main,title={Глоссарий}]
    \newpage
    \section{A/B тестирование: обзор и постановка задачи}
    \subfile{subfiles/analysis.tex}
    \newpage
    \section{Спецификации системы}
    \subfile{subfiles/specifications.tex}
    \newpage
    \section{Подробности реализации и результаты внедрения}
    \subfile{subfiles/implementation.tex}
    \newpage
    \section*{Заключение}
    \subfile{subfiles/conclusion.tex}
    \newpage
    \printbibliography
    \newpage
    \begin{supervisorreview}[Поликутина Евгения Юрьевича]{отлично}{90}{}
    	ОТЗЫВ
    \end{supervisorreview}
    \begin{review}[Поликутина Евгения Юрьевича]{отлично}{}
    	РЕЦЕНЗИЯ
    \end{review}
\end{document}

